% FIXED
\section{AAS Algorithms}
\label{s:aas:algorithms}
Here, we give detailed algorithms that implement our AAS from \cref{s:aas}.
Recall that our AAS is just a forest of BPTs with corresponding BFTs.
In particular, observe that each forest node has a BPT accumulator associated with it, while root nodes in the forest have BFTs associated with them.
Our algorithms described below operate on this forest, adding new leaves, merging nodes in the forest and computing BFTs in the roots.

\parhead{Trees notation.}
\label{s:prelim:notation:trees}
The $|$ symbol denotes string concatenation.
A \textit{tree} is a set of nodes denoted by binary strings in a canonical way.
The root of a tree is denoted by the empty string $\varepsilon$ and the left and right children of a node $w$ are denoted by $w|0$ and $w|1$ respectively.
If $b\in\{0,1\}$, then the sibling of $w = v|b$ is denoted by $\sibling(w) = v|\overline{b}$, where $\overline{b} = 1-b$.
A \emph{path} from one node $v$ to its ancestor node $w$ is denoted by $\treepath[v,w] = \{u_1 = v, u_2 = \parent(u_1), \dots, u_\ell = \parent(u_{\ell-1}) = w\}$.
The parent node of $v = w|b$ is denoted by $\parent(v) = \parent(w|b) = w$.
We also use $\treepath[v,w) = \treepath[v,w]-\{w\}$.

\parhead{Forest notation.}
\label{s:prelim:notation:forests}
Let $F_i$ denote a forest of up to $\beta$ leaves that only has $i$ leaves in it (e.g., see \cref{f:forest}).
Intuitively, a forest is a set of trees where each tree's size is a \textit{unique} power of two (e.g., see $F_5$ in \cref{f:forest}).
The unique tree sizes are maintained by constantly merging trees of the same size.
% When \beta = 1, there will be one leaf with 0 bits (\ceil(\log(1)) = 0): the empty string.
% When \beta = 2, we need \ceil(\log(2)) = 1 bit for each of the two leaves, and their root will be the empty string, and so on.
% When \beta = 3, we need \ceil(\log(3)) = 2 bits for each of the three leaves.
Let $\bin^\beta(x)$ denote the $\ceil{\log{\beta}}$-bit binary expansion of a number $x$ (e.g., $\bin^{14}(6)=0110$).
(Note that $\bin^1(x) = \varepsilon,\forall x$ because $\ceil{\log{1}}=0$.)
In our AAS, $\bin^\beta(i)$ denotes the $i$th inserted leaf, where $i$ starts at 0 (e.g., see leaves 000 through 111 in $F_5$ in \cref{f:forest}).
Let $\roots(F_i)$ denote all the roots of all the trees in the forest (e.g., $\roots(F_5) = \{0, 111\}$ in \cref{f:forest}).
Let $\leaves(F_i)$ denote all the leaves in the forest (e.g., $\leaves(F_3) = \{000, 001, 010\}$ in \cref{f:forest}).

\parhead{AAS notation.}
Note that \textbf{assert$(\cdot)$} ensures a condition is true or fails the calling function otherwise. 
Let $\dom(f)$ be the domain of a function $f$. %(i.e. $\forall x \in \dom(f), \exists y$ such that $f(x) = y$).
We use $f(x)=\bot$ to indicate $x\notin \dom(f)$.
Let $\AS_i$ denote our AAS with $i$ elements.
Each node $w$ in the forest stores extractable accumulators $\acc_w, \eacc_w$ of its BPT together with a Merkle hash $\hash_w$.
Internal nodes (i.e., non-roots) store a subset proof $\pi_w$ between $\acc_w$ and $\acc_{\parent(w)}$.
The digest $d_i$ of $\AS_i$ maps each root $r$ to its Merkle hash $\hash_r$.
Every root $r$ stores a disjointness proof $\disj_r$ between its BPT and BFT.
For simplicity, we assume server algorithms implicitly parse out the \myblue{\textbf{bolded blue variables}} from $\AS_i$.
% The algorithms below sometimes incrementally build a function $f$ by adding new points $\left(u, v = f(u)\right)$ to its graph, either via $f = f\cup \{(u,v)\}$ or via $f(u) \gets v$.

\parhead{Server algorithms.}
{\setup}$(\cdot)$ generates large enough $q$-PKE public parameters $\mathcal{PP}_q(s,\tau)$ (see \cref{d:q-pke}), given an upper bound $\beta$ on the number of elements.
Importantly, the server forgets the trapdoors $s$ and $\tau$ used to generate the public parameters.
In other words, this is a \emph{trusted setup} phase (see \cref{s:discussion}).

% Assuming elements are hashed to 2\lambda bits:
%
% The size of the AAS server public parameters
% --------------------------------------------
% \beta > 0 is the max # of elements
% When \beta = 2^k:
%  - the worst-case forest has a single BPT with \beta leaves, one for each element
%  - each element has 2\lambda bits and thus 2\lambda + 1 prefixes (b.c. we include the empty prefix \varepsilon, which we could exclude actually, since all elements have it)
%  - so, the root node of the BPT accumulates (2\lambda + 1)\beta prefixes (this is only true for \beta = 1 because, when \beta > 1, some prefixes will coincide)
% When 2^k < \beta < 2^{k+1}:
%  - the worst-case forest's largest BPT has \ell = 2^{\floor{\log{\beta}}} leaves
%    (e.g., If \beta = 10, the largest BPT will have 8 leaves, and the next BPT will have 2 leaves.
%     Incidentally, when \beta is not a power of two, we actually support adding more than \beta elements: 2^{k+1} - 1 to be exact.
%     For \beta = 10, we can add 8 + 4 + 2 + 1 = 15.)
%  - so, the root node of the BPT accumulates (2\lambda+1)\ell prefixes
%
% So, in general, we need (2\lambda+1)\ell prefixes in the root node of the largest BPT, where \ell = 2^{\floor{\log{\beta}}}
% Since a polynomial $\sum_{i=1}^q (x-r_i)$ with $q$ roots has degree $q$, that means the degree of the root BPT accumulator will be (2\lambda+1)\ell.
% (Note: In the implementation, we could avoid accumulating the empty prefix, since all keys will have it. But we do not.)
%
% The size of the AAS clients public parameters
% ---------------------------------------------
% The client needs to be able to reconstruct any leaf accumulato over 2\lambda+1 prefixes.
% So he needs to commit to polynomials of degree exactly $q=2\lambda + 1$

\begin{algorithm}[H]
    \footnotesize
    \begin{algorithmic}[1]
    \caption{\small Computes public parameters (trusted setup)}
    \label{a:aas:setup}
    \Function{\setup}{$1^\lambda, \beta$} $\rightarrow (pp, VK)$ \Comment{Generates $q$-PKE public parameters}
        \State
            $\ell \gets 2^{\floor{\log{\beta}}} \qquad
             q \gets (2\lambda + 1)\ell \qquad
             ( \Group,\GT,p,g,e(\cdot,\cdot) ) \leftarrow \mathcal{G}(1^\lambda)$
        \State
            $s\stackrel{\$}{\gets} \Fp \qquad
             \tau\stackrel{\$}{\gets} \Fp \qquad
             VK=((g^{s^i})_{i=0}^{2\lambda + 1}, g^\tau)$
        \State \Return $((( \Group,\GT,p,g,e(\cdot,\cdot) ), \beta, \mathcal{PP}_q(s,\tau)), VK )$
    \EndFunction
    \end{algorithmic}
\end{algorithm}

{\append}$(\cdot)$ creates a new leaf $\ell$ for the element $k$ (\crefrange{a:aas:append:create-leaf-begin}{a:aas:append:create-leaf-end}).
Recursively merges equal-sized BPTs in the forest, as described in \cref{s:aas:construction} (\crefrange{a:aas:append:merge-begin}{a:aas:append:merge-end}).
In this process, computes subset proofs between old BPT roots and the new BPT.
Merging ends when the newly created BPT $w$ has no equal-sized BPT to be merged with.
Recall from \cref{s:prelim:polycommit} that $\Hp$ maps elements to be accumulated to field elements in $\Fp$.
\begin{algorithm}[H]%[tb] % sigle column
    \caption{\small Appends a new $i$th element to the AAS, $i\in[0,\beta-1]$}
    \label{a:aas:append}
    \footnotesize
    \begin{algorithmic}[1]
    \Function{\append}{$pp,\AS_i, d_i, k$} $\rightarrow (\AS_{i+1}, d_{i+1})$
        \State $w \gets \bin^\beta(i) \qquad \elems_w \gets \{k\}$\Comment{Create new leaf $w$ for element $k$}
        \label{a:aas:append:create-leaf-begin}
        \State $(\accpoly_w, {\acc}_w, \cdot) \gets \accumulate(P(\elems_w)) \quad \hash_w \gets \Hb(w|\bot|{\acc}_w|\bot)$
        \label{a:aas:append:create-leaf-end}

        \LineComment{``Merge'' old BPT roots with new BPT root (recursively)}
        \While{$\sibling(w)\in \roots(F_i)$}
        \label{a:aas:append:merge-begin}
            \State $\ell \gets\sibling(w) \qquad p\gets \parent(w) \qquad \elems_p \gets \elems_\ell\cup \elems_w$
            \State $(\accpoly_p, {\acc}_p, \eacc_p) \gets \accumulate(P(\elems_p))\qquad \hash_p = \Hb(p|\hash_\ell|{\acc}_p|\hash_w)$
            \State $(\cdot, \subsetProof_\ell, \cdot) \gets \accumulate(P(\elems_p\setminus \elems_\ell))$
            \State $(\cdot, \subsetProof_w, \cdot) \gets \accumulate(P(\elems_p \setminus \elems_w)) \qquad w\gets p$
        \EndWhile
        \label{a:aas:append:merge-end}
        % Note: Adding a single leaf will always create exactly one new root, possibly by merging more than two old roots.

        \LineComment{Invariant: $w$ is a new root in $F_{i+1}$. Next, computes $w$'s frontier.}
        \label{a:aas:append:root-at-begin}
        \State $(\fropoly_w,\bft_w) \gets \createfrontier(F(\elems_w))$
        % We need frontier polynomial \fropoly_w here to compute the EEA coeffs below!
        \State $(y,z)\gets \mathsf{ExtEuclideanAlg}(\accpoly_w, \fropoly_w)\qquad {\disj}_w \gets ( g^{y(s)}, g^{z(s)})$
        \label{a:aas:append:root-at-end}

        \State Store updated AAS state (i.e., the \myblue{\textbf{bolded blue}} variables) into $\AS_{i+1}$
        \State $d_{i+1}(r) \gets \hash_r, \forall r \in \roots(F_{i+1})$\Comment{Set new digest}
        \State \Return $\AS_{i+1}, d_{i+1}$
    \EndFunction

    % Note that this function is implicitly given access to all the public parameters it needs.
    \Function{\accumulate}{$T$}
        \State \Return $(\accpoly, g^{\accpoly(s)}, g^{\tau \accpoly(s)})$ where $\accpoly(x)=\prod_{w \in T} {(x-\Hp(w))}$
    \EndFunction
    \end{algorithmic}
\end{algorithm}

If $k$ is in the set, $\provememb(\cdot)$ sends a Merkle path to $k$'s leaf in some tree with root $r$ (\crefrange{a:aas:provememb:paths-begin}{a:aas:provememb:paths-end}) via $\provepath(\cdot)$ (see \cref{a:aas:provepath}).
This path contains subset proofs between every node's accumulator and its parent node's accumulator.
If $k$ is not in the set, then $\provememb(\cdot)$ sends frontier proofs in each BFT in the forest (\crefrange{a:aas:provememb:frontier-begin}{a:aas:provememb:frontier-end}) via $\provefrontier(\cdot)$ (see \cref{a:at:provefrontier}).

\begin{algorithm}[H]%[tb] % sigle column
    \caption{\small Constructs a (non)membership proof}
    \label{a:aas:provememb}
    \label{a:aas:provepath}
    \footnotesize
    \begin{algorithmic}[1]
    \Function{\provememb}{$pp,\AS_i,k$} $\rightarrow (b,\pi)$
        \State Let $\ell\in \leaves(F_i)$ be the leaf where $k$ is stored or $\bot$ if $k\notin \AS_i$
        \If{$k \in \AS_i$} \Comment{Construct Merkle path to element}
            \label{a:aas:provememb:paths-begin}
            \State Let $r\in \roots(F_i)$ be the root of the tree where $k$ is stored
            \State $\pi \gets \provepath(\AS_i, \ell, r,\bot)\qquad b\gets 1$ % \qquad R\gets R-\{r\}
            \label{a:aas:provememb:paths-end}
        \Else \Comment{Prove non-membership in all BFTs}
            \label{a:aas:provememb:frontier-begin}
            \State $\CP_r\gets \provefrontier(\AS_i,r,k),\forall r\in \roots(F_i)$
            % We don't need extractability for BPT accumulators in non-membership proofs
            % We do need the hashes of the root's children (if any) for the client to verify this BPT accumulator against the Merke root hash
            % (note if r|c does not exist because r is a leaf, then we assume h_{r|c} equals \bot)
            \State $\pi\gets \proverootaccs(\AS_i, \pi)\quad b\gets 0$
            \label{a:aas:provememb:frontier-end}
        \EndIf
        \State \Return $b, ( \ell, \pi, (\CP_r)_{r\in \roots(F_i)}, (\disj_r)_{r\in \roots(F_i)})$
    \EndFunction

    % Includes leaf (u) and root (r) BPT accumulators, as well as (u)'s subset proof
    \Function{\provepath}{$\AS_i,u,r,\pi$} $\rightarrow \pi$\Comment{Precondition: $r$ is a root in $F_i$}
        \State $\pi(r) \gets ( \bot, \acc_r, \eacc_r, \bot)$
        \LineComment{Overwrites $\pi(w)$ set by previous \provepath call (if any)}
        \State $\pi(w) \gets ( \bot, \acc_w, \eacc_w, \subsetProof_w), \forall w\in \treepath[u,r)$
        \LineComment{Only sets $\pi(\sibling(w))$ if not already set from previous \provepath call!}
        \For{$w\in \treepath[u,r)$ where $\sibling(w)\notin \dom(\pi)$}
            \State $\pi(\sibling(w)) \gets ( \hash_{\sibling(w)}, \bot,\bot,\bot)$
        \EndFor
        \State \Return $\pi$
    \EndFunction

    \Function{\proverootaccs}{$\AS_i,\pi$} $\rightarrow \pi$
        % Note: Non-membership proofs don't need the extractable counterpart, but \verappendonly does.
        \State $\pi(r)\gets (\bot,\acc_{r}, \eacc_{r}, \bot), \forall r\in\roots(F_i),$
        \State $\pi(r|c)\gets (\hash_{r|c},\bot,\bot,\bot), \forall r\in\roots(F_i), \forall c\in \{0,1\}$
    \EndFunction
    \end{algorithmic}
\end{algorithm}

For each root $r$ in $F_i$, {\proveappendonly}$(\cdot)$ sends a Merkle path to an ancestor root in $F_j$, if any.
The Merkle path contains subset proofs between all BPT accumulators along the path.
It also contains the root BPT accumulators from $F_i$, which the client will verify against his digest $d_i$.

\parhead{Client algorithms.}
{\verappendonly}$(\cdot)$ first ensures that $d_i$ and $d_j$ are digests at version $i$ and $j$ respectively (\crefrange{a:aas:verappendonly:check-digest-begin}{a:aas:verappendonly:check-digest-end}).
Before checking subset proofs, {\verappendonly}$(\cdot)$ validates the old root BPT accumulators in $\pi_{i,j}$ against the Merkle roots in $d_i$ (\crefrange{a:aas:verappendonly:check-old-root-accs-begin}{a:aas:verappendonly:check-old-root-accs-end}).
Then, checks that each root $r$ from $F_i$ is a subset of some root in $F_j$ by checking subset proofs (\cref{a:aas:verappendonly:paths}) via $\verpath(\cdot)$ (see \cref{a:helper:verpath}).
{\verappendonly}$(\cdot)$ enforces fork-consistency implicitly when verifying Merkle hashes.

\begin{algorithm}[H]%[tb] % sigle column
    \caption{\small Creates and verifies append-only proofs}
    \label{a:aas:proveappendonly}
    \label{a:aas:verappendonly}
    \footnotesize
    \begin{algorithmic}[1]
    \Function{\proveappendonly}{$pp,\AS_i,\AS_j$} $\rightarrow \pi$
        \If{$\roots(F_i)\subset\roots(F_j)$}
            \Return $\bot$
        \EndIf
        \State Let $R=\{$roots $\in F_i$ but $\not\in F_j\}$ and $r'\in \roots(F_j)$ be their ancestor root
        % Note: When adding two paths, appends to proof \pi or even overwrites it
        \State $\pi\gets \provepath(\AS_j, r,r',\pi), \forall r \in R\quad \pi\gets\proverootaccs(\AS_i,\pi)$
        \State \Return $\pi$
    \EndFunction

    \Function{\verappendonly}{$VK, d_i, i, d_j, j, \pi_{i,j}$} $\rightarrow \{T,F\}$
        % Version of digest is implicitly given by the node numbering fixed through Merkle hashing
        % For the fork-consistency definition, VerAppendOnly needs to check the old version of the digest too (technicality).
        \Assert $d_i(r) \ne \bot \Leftrightarrow r\in \roots(F_i)$ \Comment{Is valid version $i$ digest?}
        \label{a:aas:verappendonly:check-digest-begin}
        \Assert $d_j(r) \ne \bot \Leftrightarrow r\in \roots(F_j)$ \Comment{Is valid version $j$ digest?}
        \label{a:aas:verappendonly:check-digest-end}
        \Assert $\forall r \in \roots(F_i) \cap \roots(F_j), d_i(r)=d_j(r)$
        
        \State Let $R=\{$roots $\in F_i$ but $\not\in F_j\}$ \Comment{i.e., old roots with paths to new root}
        % Recall that d_i(r) only stores the Merkle root hash h_r and not the BPT accumulator a_r.
        % The proof \pi_{i,j} will give a_r but it could give a completely different one, breaking the append-only property.
        % Thus, whatever it gives must be verified.
        \ForAll{$r\in \roots(F_i)$} \Comment{Check proof gives correct old root accumulators}
            \label{a:aas:verappendonly:check-old-root-accs-begin}
            \State $(\cdot,a_{r},\cdot,\cdot)\gets \pi(r)$ \quad $({h}_{r|b},\cdot,\cdot,\cdot) \gets \pi(r|b),\forall b\in \{0,1\}$
            \Assert $d_i(r) = \Hb(r|h_{r|0}|a_r|h_{r|1})$
        \EndFor
        \label{a:aas:verappendonly:check-old-root-accs-end}
        \State $\forall r\in R$, fetch $h_r$ from $d_i(r)$ and update $\pi_{i,j}(r)$ with it
        \Assert $\pi_{i,j}$ is well-formed Merkle proof for all roots in $R$
        \Assert $\forall r\in R,\verpath(d_j, r, \pi_{i,j})$
        \label{a:aas:verappendonly:paths}
    \EndFunction
    \end{algorithmic}
\end{algorithm}

If $k$ is stored at leaf $\ell$ in the AAS, {\vermemb}$(\cdot)$ reconstructs $\ell$'s accumulator from $k$.
Then, checks if there's a valid Merkle path from $\ell$ to some root, verifying subset proofs along the path via $\verpath(\cdot)$ (see \cref{a:helper:verpath}).
If $k$ is not in the AAS, {\vermemb}$(\cdot)$ verifies frontier proofs for $k$ in each BFT in the forest via $\verfrontier(\cdot)$ (see \cref{a:at:verfrontier}).

\begin{algorithm}[H]%[tb] % single column
    \caption{\small Verifies a (non)membership proof}
    \label{a:aas:vermemb}
    \label{a:helper:verpath}
    \footnotesize
    \begin{algorithmic}[1]
    \Function{\vermemb}{$VK, d_i,k,b,\pi_k$} $\rightarrow \{T,F\}$
    \State Parse $\pi_{k}$ as $\ell, \pi, (\CP_r)_{r\in \roots(F_i)}, (y_r, z_r)_{r\in \roots(F_i)}$
    \If{$b = 1$} \Comment{This is a membership proof being verified}
    \label{a:aas:vermemb:foreach-leaf}
        \label{a:aas:vermemb:verpath-begin}
        \State $(\cdot, a_\ell, \hat{a}_\ell)\gets \accumulate(P(\{k\}))\qquad h_\ell \gets \Hb(\ell|\bot|a_\ell|\bot)$
        % VerPath will need a_\ell when recursively checking the append-only property
        % Also, this is an update, not an overwrite, because we have to preserve the subset proof from \pi(\ell) (if any)
        \State Update $\pi(\ell)$ with $h_\ell$ and accumulators $a_\ell$ and $\hat{a}_\ell$
        \Assert $\pi$ is well-formed Merkle proof for leaf $\ell \wedge {\verpath(d_i, \ell, \pi)}$
        \label{a:aas:vermemb:verpath-end}
    \Else  \Comment{This is a non-membership proof being verified}
        \label{a:aas:vermemb:verfrontier-begin}
        % Get BFT and BPT accs. Validate BPT accs against digest.
        \ForAll{$r\in \roots(F_i)$} \Comment{Check BFTs}
            \State $(\cdot,a_{r},\cdot,\cdot)\gets \pi(r)$ \quad $(o_r, \cdot) \gets \CP_r(\varepsilon)$
            \State $({h}_{r|b},\cdot,\cdot,\cdot) \gets \pi(r|b),\forall b\in \{0,1\}$
            \Assert $d_i(r) = \Hb(r|h_{r|0}|a_r|h_{r|1})$
            \Assert $e(a_r,y_r)e(o_r,z_r) = e(g,g) \wedge \verfrontier(k, \CP_r)$
        \EndFor
        \label{a:aas:vermemb:verfrontier-end}
    \EndIf
    \EndFunction

    % Note: w is always a node in F_i
    \Function{\verpath}{$d_k,w,\pi$} $\rightarrow \{T,F\}$
        \State Let $r\in \roots(F_k)$ denote the ancestor root of $w$
        \LineComment{Walk path invariant: $u$ is \textit{not} a root node (but $\parent(u)$ might be)}
        \For{$u\gets w; u \neq r; u\gets\parent(u)$}
            \State $p \gets \parent(u)$\Comment{Check subset proof and extractability (below)}
            \State $(\cdot, a_{u}, \hat{a}_{u}, \pi_{u}) \gets \pi(u)\quad(\cdot, a_{p}, \hat{a}_{p}, \cdot) \gets \pi(p)$
            \Assert $e(a_u, \pi_u) = e(a_p, g) \wedge e(a_u, g^\tau) = e(\hat{a}_u, g)$
            \label{a:aas:verpath:extractability-check}
        \EndFor
        \Assert $d_k(r) = \mathsf{MerkleHash}(\pi, r)$ \Comment{Invariant: $u$ equals $r$ now}
        \label{a:aas:verpath:merklehash}
        \Assert $e(a_r, g^\tau) = e(\hat{a}_r, g)$\Comment{Is root accumulator extractable?}

    \EndFunction
    % Note: Does NOT modify Merkle proof \pi. Just returns the computed Merkle root hash.
    \Function{$\mathsf{MerkleHash}$}{$\pi, w$} $\rightarrow h_w$ \Comment{Precondition: $\pi$ is well-formed proof}
    \label{a:aas:merklehash}
        \State $(h_w, a_w, \cdot,\cdot) \gets \pi(w)$
        \If{$h_w = \bot$}
            \State \Return $\Hb(w|\mathsf{MerkleHash}(\pi, w|0)|a_w|\mathsf{MerkleHash}(\pi, w|1))$
        \Else
            \State \Return $h_w$
        \EndIf
    \EndFunction
    \end{algorithmic}
\end{algorithm}

\parhead{Frontier algorithms.}
{\createfrontier}$(\cdot)$ creates a BFT level by level, starting from the leaves, given a set of frontier prefixes $F$.
Given a key $k\notin \AS_i$ and a root $r$, {\provefrontier}$(\cdot)$ returns a frontier proof for $k$ in the BFT at root $r$.
{\verfrontier}$(\cdot)$ verifies a frontier proof for one of $k$'s prefixes against a specific root BFT accumulator.
\begin{algorithm}[H]
    \begin{algorithmic}[1]
    \caption{\small Manages BFT of a set}
    \label{a:at:createfrontier}
    \label{a:at:provefrontier}
    \label{a:at:verfrontier}
    \footnotesize
    \Function{\createfrontier}{$F$} $\rightarrow (\phi,\sigma)$
        \State $i\gets 0\qquad S_w\gets \varnothing, \forall w$
        \For{$\rho \in F$} \Comment{First, build BFT leaves, with $g^{s-\Hp(\rho)}$ for each prefix $\rho$}
            \State $w\gets \bin^{|F|}(i)\qquad S_w \gets \rho\qquad i\gets i+1$
            \State $(\fropoly_w,\fac,\efac) \gets \accumulate(S_w)\quad \sigma(w) \gets (\fac, \efac)$
        \EndFor

        % Example: When |F| = 5, this code merges the first two leaves, then the next two leaves, then for the 5th leaf without a sibling, it just "merges" it with its empty sibling.
        % In other words, its parent will have the same frontier accumulator and the tree looks like this:
        %        r
        %    -       -
        %  *   *   *
        % 1 2 3 4 5
        % In our C++ implementation, we have a more efficient implementation, where the tree looks like this:
        %        r
        %    -       5
        %  *   *
        % 1 2 3 4
        \For{$i \gets \ceil{\log{|F|}}; i\ne 0; i\gets i-1$} \Comment{Then, build BFT level by level}
            \State $j\gets 0\qquad \mathsf{levelSize} \gets 2^i\qquad u\gets \bin^{\mathsf{levelSize}}(0)$
            \While{$S_u \ne \varnothing$} \Comment{Merge sibling accumulators on level $i$}
                \State $p \gets \parent(u)\quad S_p \gets S_u\cup S_{\sibling(u)} \quad j \gets j+2$
                \State $(\fropoly_p,\fac,\efac) \gets \accumulate(S_p)\quad \sigma(p) \gets (\fac, \efac)\quad u \gets \bin^\mathsf{levelSize}(j)$
            \EndWhile
        \EndFor
        \State \Return $( \fropoly_\varepsilon, \sigma)$
    \EndFunction

    \Function{\provefrontier}{$\AS_i, r, k$} $\rightarrow \CP$
        \State Let $\rho$ be the smallest prefix of $k$ that is not in $P(\elems_r)$
        \State Let $\ell$ denote the leaf where $\bft_r(\ell) = g^{(s-\Hp(\rho))}$
        \State $\CP(\varepsilon) \gets \bft_r(\varepsilon)$ \Comment{Copy root BFT accumulator}
        \For{$w\in \treepath[\ell, \varepsilon)$}\Comment{Copy path to $\rho$'s BFT leaf}
            \State $\CP(w) \gets \bft_r(w)$
            \If{$\bft_r(\sibling(w))\ne \bot}$
                % Note: Siblings need not be extractable, but for simplicity of the code we don't optimize.
                % Asymptotically proof-size remains the same.
                \State $\CP(\sibling(w)) \gets \bft_r(\sibling(w))$
            \Else
                % This handles frontier tree sizes that are not powers of two.
                % Continuing with the |F|=5 example from \createfrontier, the 5th leaf (at level 3) would have an empty sibling.
                % So would its parent at level 2. But its parent at level 1 wouldn't. And level 0 is the root. Just look at the tree:
                %
                %        r
                %    -       -
                %  *   *   *
                % 1 2 3 4 5
                \State $\CP(\sibling(w)) \gets (g, g^\tau)$
            \EndIf
        \EndFor
        \State \Return $\CP$
    \EndFunction

    \Function{\verfrontier}{$k,\CP$} $\rightarrow \{T,F\}$
        \LineComment{Find leaf $\ell$ in $\CP$ with a prefix $\rho$ for $k$, or fail.}
        \Assert $\exists \ell,\exists\rho$ s.t. $\rho\in P(\{k\}) \wedge g^{(s-\Hp(\rho))} = \CP(\ell)$
        \Assert $e(\fac, g^\tau) = e(\efac_w, g)$ where $(\fac,\efac)\gets \CP(\varepsilon)$
        %\Comment{Root BFT accumulator is extractable?}
        % NOTE: For loop below implicitly verifies against root in BFT as well!
        \For{$w\in \treepath[\ell, \varepsilon)$}\Comment{Verify $\rho$'s membership in the BFT}
            \State $(c_w,\hat{c}_w) \gets \CP(w)\quad (s_w,\cdot)\gets \CP(\sibling(w))$
            \State $(p_w,\cdot) \gets \CP(\parent(w))$
            \Assert $e(c_w, s_w) = e(p_w, g)\wedge e(c_w, g^\tau) = e(\hat{c}_w, g)$
        \EndFor
    \EndFunction
    \end{algorithmic}
\end{algorithm}

\begin{theorem}
    \label{thm:aas}
    Under the $q$-SBDH and $q$-PKE assumptions, and assuming that $\Hb$ is a secure CRHF, our construction is a secure AAS as per ~\cref{d:secure-aas-definition}.
\end{theorem}

We prove \cref{thm:aas} in \cref{s:aas:proofs:membership-security}.
