% FIXED
%
% The "author" command and its associated commands are used to define the authors and their affiliations.
% Of note is the shared affiliation of the first two authors, and the "authornote" and "authornotemark" commands
% used to denote shared contribution to the research.
%
\author{Alin Tomescu}
%\email{alinush@mit.edu}
\affiliation{%
  \institution{Massachusetts Institute of Technology}
}
%\orcid{1234-5678-9012}
\author{Vivek Bhupatiraju}

\affiliation{%
  \institution{Lexington High School}
  \institution{MIT PRIMES}
}
%
\author{Dimitrios Papadopoulos}
\affiliation{%
  \institution{Hong Kong University of Science and Technology}
}
\author{Charalampos Papamanthou}
\affiliation{%
  \institution{University of Maryland}
}
%
\author{Nikos Triandopoulos}
\affiliation{%
 \institution{Stevens Institute of Technology}
}
% 
\author{Srinivas Devadas}
\affiliation{%
  \institution{Massachusetts Institute of Technology}
}

%
% By default, the full list of authors will be used in the page headers. Often, this list is too long, and will overlap
% other information printed in the page headers. This command allows the author to define a more concise list
% of authors' names for this purpose.
\renewcommand{\shortauthors}{Tomescu et al.}

% NOTE: Go to main.tex if you want to have "Anonymous authors"
